% Pour un exemple de vignette, voir
% browseVignettes("network")

%\documentclass{article}
\documentclass[article,nojss]{jss}
% ou documentclass{jss} pour le Journal of Statistical Software

\usepackage[utf8]{inputenc}
\usepackage[T1]{fontenc}
% RJournal
%\usepackage{RJournal}
\usepackage{amsmath,amssymb,array}
%\usepackage{booktabs}

\usepackage{Sweave}

%Les lignes suivantes doivent rester commentées, pour être utilisée par R et pas par LaTeX
% !Rnw weave = Sweave
%!\SweaveUTF8
%\VignetteIndexEntry{About the stemmatology package}
%\VignetteIndexEntry{stemmatology Vignette}


\author{J.B. Camps, Florian Cafiero}

\title{\pkg{stemmatology}: A \proglang{R} Stemmatology Package}
\Plaintitle{stemmatology: A R Stemmatology Package}
\Shorttitle{\pkg{stemmatology}: A \proglang{R} Stemmatology Package}

\Abstract{
This package includes various functions for stemmatological analysis. Its purpose is to provide the user with implementation of the PCC Method, but also with various other tools and utilities...
}

\Keywords{stemmatology, network, graphs, relational data}
%\Keywords{relational data, data structures, graphs, \pkg{network}, \pkg{statnet}, \proglang{R}}

\Address{
JB Camps\\
École nationale des chartes\\
\url{jbcamps@hotmail.com}\\
URL: \url{...}
}

\begin{document}

\definecolor{Sinput}{rgb}{0.19,0.19,0.75}
\definecolor{Soutput}{rgb}{0.2,0.3,0.2}
\definecolor{Scode}{rgb}{0.75,0.19,0.19}
\DefineVerbatimEnvironment{Sinput}{Verbatim}{formatcom = {\color{Sinput}}} 
\DefineVerbatimEnvironment{Soutput}{Verbatim}{formatcom = {\color{Soutput}}}
\DefineVerbatimEnvironment{Scode}{Verbatim}{formatcom = {\color{Scode}}} 
\renewenvironment{Schunk}{}{}

\Sconcordance{concordance:stemmatology.tex:stemmatology.Rnw:%
1 143 1 1 4 3 0 1 3 2 0 1 2 1 0 1 3 1 0 1 2 1 0 1 2 1 0 1 1 1 2 1 0 1 1 %
1 2 4 0 1 2 2 1}


% Ce fichier sert à donner une documentation généraliste du package sous la forme d'un article, qui peut être plus complet que la documentation proprement dite des fichiers .Rd
% Voir notamment http://www.stats.uwo.ca/faculty/murdoch/ism2013/5Vignettes.pdf
% ainsi que http://cran.r-project.org/doc/manuals/R-exts.html#Documenting-packages

\maketitle

%\begin{abstract}
%This package includes various functions for stemmatological analysis. Its purpose is to provide the user with ...
%\end{abstract}
\pkg{Stemmatology} in 

\section*{Requirements}

\begin{Schunk}
\begin{Sinput}
> library(network)
\end{Sinput}
\end{Schunk}


\section{Functions for the PCC method}



\section{Import functions}

\section{Various exploratory methods}

\section{Other Methods}

\section{Tools}

\section{Low level functions}



\end{document}
